 \documentclass[a4paper,11pt]{article}

\usepackage{amsmath}
\usepackage{amssymb}
%\usepackage{amsthm}
\usepackage{graphicx}
\usepackage{epstopdf}
\epstopdfsetup{update}

\newcommand{\ba}{\begin{array}}
\newcommand{\ea}{\end{array}}

\newcommand{\bea}{\begin{eqnarray}}
\newcommand{\eea}{\end{eqnarray}}

\newcommand{\bc}{\begin{center}}
\newcommand{\ec}{\end{center}}

\newcommand{\ds}{\displaystyle}

\newcommand{\bt}{\begin{tabular}}
\newcommand{\et}{\end{tabular}}

\newcommand{\bi}{\begin{itemize}}
\newcommand{\ei}{\end{itemize}}

\newcommand{\bd}{\begin{description}}
\newcommand{\ed}{\end{description}}

\newcommand{\bp}{\begin{pmatrix}}
\newcommand{\ep}{\end{pmatrix}}

\newcommand{\pd}{\partial}
\newcommand{\sech}{\mbox{sech}}

\newcommand{\cf}{{\it cf.}~}

\newcommand{\ltwo}{L_{2}(\mathbb{R}^{2})}
\newcommand{\smooth}{C^{\infty}_{0}(\mathbb{R}^{2})}

\newcommand{\br}{{\bf r}}
\newcommand{\bk}{{\bf k}}
\newcommand{\bv}{{\bf v}}

\newcommand{\gnorm}[1]{\left|\left| #1\right|\right|}
\newcommand{\ipro}[2]{\left<#1,#2 \right>}

%\setlength{\topmargin}{-40pt} \setlength{\oddsidemargin}{0pt}
%\setlength{\evensidemargin}{0pt} \setlength{\textwidth}{460pt}
%\setlength{\textheight}{680pt}
\title{Evolution of Narrow-Band Spectra in Deep-Water Constant Vorticity Flows}
\author{Christopher W. Curtis and Mackensie Murphy}
\date{}
\begin{document}
\maketitle
\section*{Introduction}

Building off the now seminal work in \cite{alber}, there is a relatively wide range of literature which shows that nonlinear instabilities, in particular the modulational instability (MI), are responsible for significant modifications to the statistical properties of water waves; see \cite{dysthe2,dysthe3,onorato,thomas2012nonlinear} among others.   These results have potentially significant impact on the understanding of rogue wave formation among other oceanographic phenomena.  However, aside from 

To describe the evolution of a nearly linear wave-train in a constant-vorticity-shear current running over infinitely deep water, we use the form of the nonlinear Schr\"{o}dinger equation derived in \cite{curtis8}, though see also \cite{thomas2012nonlinear}, given by 
\[
i\pd_{\tau}\eta_{1} + \alpha_{d}\pd_{\xi}^{2}\eta_{1} + \alpha_{nl}\left|\eta_{1} \right|^{2}\eta_{1} = 0, 
\]
where $\tau = \epsilon^{2}t$, $\xi = \epsilon(x+c_{g}t)$, and 
\begin{align*}
c_{g} = & \frac{1+3\tilde{\sigma}k_{0}^{2}}{2s\Omega - \omega}\\
\alpha_{d}(k_{0},\omega) = & \frac{(c^2_{g} - 3|k_{0}|\tilde{\sigma})}{2\Omega-s\omega},\\
\alpha_{nl}(k_{0},\omega) = & \frac{k_{0}\left( sk_{0}^{3}\left(8 + \tilde{\sigma}k_{0}^{2} + 2(\tilde{\sigma}k_{0}^{2})^{2}\right) + \omega \alpha_{v}\right)}{\left(2s\Omega -\omega\right)(1+c_{g}\omega)\left(4\Omega^2-s(2k_{0}(1+4\tilde{\sigma}k_{0}^{2})+2\omega\Omega)\right)},
\end{align*}
where $s = \mbox{sgn}(k_{0})$, $\Omega$ denotes the dispersion relationship, and $\omega$ is the non-dimensionalized magnitude of the vorticity of the flow, and $\tilde{\sigma}$ is the surface tension.  In a now seminal paper, \cite{alber} studies families of profiles with relatively narrow banded spectra around $k_{0}$ and modulational instability (MI) is either manifested or supressed.  The results in \cite{alber} were confirmed numerically in \cite{dysthe2,dysthe3} by examing the mean properties associated with ensembles of initial conditions.  This was done by starting from the $2L$ periodic-initial condition
\[
\eta_{1}(\xi,0) = \epsilon_{rms}\sqrt{\frac{\delta \tilde{k}}{\sigma\sqrt{\pi}}} \sum_{k=-K+1}^{K}e^{-\tilde{k}^{2}/2\sigma^{2}}e^{i\theta_{k}}e^{i\tilde{k}\xi}, ~ \tilde{k} = \frac{\pi k}{L}
\]
where $\delta \tilde{k}=\pi/L$ and the phases $\theta_{k}$ are randomly chosen uniformly between $0$ and $2\pi$.  We can then readily show that, for $L\gg 1$, 
\[
\epsilon_{rms} \approx \overline{\left|\eta_{1}\right|^{2}}^{1/2},
\]
where $\overline{()}$ denotes the ensemble average.  In our coordinates then, the work in \cite{alber} and later confirmed in \cite{dysthe2} shows that MI is suppressed for spectral widths $\sigma$ such that 
\[
\sigma \geq \epsilon_{rms}\sqrt{\frac{2\alpha_{nl}}{\alpha_{d}}}
\]
This work has helped motivate defining this parameter as the Benjamin--Feir Index (BFI), which has been shown to be a critical parameter to understanding the statstical properties of nonlinear free-surface flows; \cite{onorato, thomas2012nonlinear}.

In part then, in this paper we establish the validity of the above bound on the spectral width for the appearance or suppression of MI.  We likewise examine the impact that vorticity plays in controlling the effective width for the onset of MI.  Moving beyond just examining the properties of the NLS equation with vorticity, we look at the statistical properties of solutions to a higher-order model, the vor-Dysthe equation derived in \cite{curtis8}.  As shown in \cite{onorato}, the higher-order terms associated with the Dysthe equation can have significant impacts on the statistics of the waves, so it is a nontrivial question to examine how the vorticity will interact with these higher-order nonlinearities.  

\bibliography{deep_water_shear_Feb2}
\bibliographystyle{unsrt}
\end{document}